\documentclass[a4paper,12pt]{article}
\usepackage{amsmath}
\begin{document}
\title{Overleaf 2}
\author{Diego Amicabile}
\date{\today}
\maketitle

\begin{equation*}
\Omega = \sum_{k=1}^{n} \omega_k
\end{equation*}

\begin{equation*}
% bad!
min_{x,y} (1-x)^2 + 100(y-x^2)^2
\end{equation*}
\begin{equation*}
% good!
\min_{x,y}{(1-x)^2 + 100(y-x^2)^2}
\end{equation*}

\begin{equation*}
\beta_i =
\frac{\operatorname{Cov}(R_i, R_m)}
{\operatorname{Var}(R_m)}
\end{equation*}

\begin{align*}
(x+1)^3 &= (x+1)(x+1)(x+1) \\
&= (x+1)(x^2 + 2x + 1) \\
&= x^3 + 3x^2 + 3x + 1
\end{align*}

Let $X_1$, $X_2$, ..., $X_n$
be a sequence of independent and identically
distributed random variables with $E$[$X_i$] = $\mu$
and $Var$[$X_j$] =$\sigma^2$ $<$ $\infty$, and let

\begin{equation*}
S_n = \frac{1}{n}{\sum_{i=1}^{n} X_i}
\end{equation*}

denote their mean. Then as $n$ approaches infinity, the random $\sqrt{n}$($S_n$-$\mu$) variables converge in distribution to a normal $N$(0,$\sigma^2$).


Let $X_1, X_2, \ldots, X_n$ be a sequence of independent and
identically distributed random variables with
$\operatorname{E}[X_i] = \mu$ and
$\operatorname{Var}[X_i] = \sigma^2 < \infty$, and let
\begin{equation*}
S_n = \frac{1}{n}\sum_{i}^{n} X_i
\end{equation*}
denote their mean. Then as $n$ approaches infinity, the
random variables $\sqrt{n}(S_n - \mu)$ converge in
distribution to a normal $N(0, \sigma^2)$.

% bonus points: the N for normal is usually set in a caligraphic
% font; you can get this using $\mathcal{N}(0, \sigma^2)$.

\end{document}
